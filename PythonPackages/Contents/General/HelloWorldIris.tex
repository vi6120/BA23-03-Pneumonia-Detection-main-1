%%%%%%%%%%%%%%%
%
% $Autor: Wings $
% $Datum: 2020-02-24 14:30:26Z $
% $Pfad: PythonPackages/Contents/General/HelloWorldIris.tex $
% $Version: 1792 $
%
% !TeX encoding = utf8
% !TeX root = PythonPackages
% !TeX TXS-program:bibliography = txs:///bibtex
%
%
%%%%%%%%%%%%%%%



% Quelle: https://machinelearningmastery.com/machine-learning-in-python-step-by-step/


\chapter{Hello World ``Iris''}

\section{Your First Machine Learning Project in Python Step-By-Step}

Do you want to do machine learning using Python, but you’re having trouble getting started?

You will complete your first machine learning project using Python.

In this step-by-step tutorial you will:

\begin{itemize}
  \item Download and install Python SciPy and get the most useful package for machine learning in Python.
  \item Load a dataset and understand it’s structure using statistical summaries and data visualization.
  \item Create 6 machine learning models, pick the best and build confidence that the accuracy is reliable.
\end{itemize}

If you are a machine learning beginner and looking to finally get started using Python, this tutorial was designed for you.

Kick-start your project with my new book Machine Learning Mastery With Python, including step-by-step tutorials and the Python source code files for all examples.

\section{How Do You Start Machine Learning in Python?}

The best way to learn machine learning is by designing and completing small projects.
    
    
\subsection{Python Can Be Intimidating When Getting Started}

Python is a popular and powerful interpreted language. Unlike R, Python is a complete language and platform that you can use for both research and development and developing production systems.
    
There are also a lot of modules and libraries to choose from, providing multiple ways to do each task. It can feel overwhelming.
    
The best way to get started using Python for machine learning is to complete a project.
    
\begin{itemize}
  \item It will force you to install and start the Python interpreter (at the very least).
  \item It will given you a bird’s eye view of how to step through a small project.
  \item It will give you confidence, maybe to go on to your own small projects.
\end{itemize}    

\subsection{Beginners Need A Small End-to-End Project}

Books and courses are frustrating. They give you lots of recipes and snippets, but you never get to see how they all fit together.
    
When you are applying machine learning to your own datasets, you are working on a project.
    
A machine learning project may not be linear, but it has a number of well known steps:
    

\begin{enumerate}
  \item Define Problem.
  \item Prepare Data.
  \item Evaluate Algorithms.
  \item Improve Results.
  \item Present Results.
\end{enumerate}

The best way to really come to terms with a new platform or tool is to work through a machine learning project end-to-end and cover the key steps. Namely, from loading data, summarizing data, evaluating algorithms and making some predictions.
    
If you can do that, you have a template that you can use on dataset after dataset. You can fill in the gaps such as further data preparation and improving result tasks later, once you have more confidence.
    
\subsection{Hello World of Machine Learning}

The best small project to start with on a new tool is the classification of iris flowers (e.g. the iris dataset).
    
This is a good project because it is so well understood.

\begin{itemize}
  \item Attributes are numeric so you have to figure out how to load and handle data.
  \item It is a classification problem, allowing you to practice with perhaps an easier type of supervised learning algorithm.
  \item It is a multi-class classification problem (multi-nominal) that may require some specialized handling.
  \item It only has 4 attributes and 150 rows, meaning it is small and easily fits into memory (and a screen or A4 page).
  \item All of the numeric attributes are in the same units and the same scale, not requiring any special scaling or transforms to get started.
\end{itemize}    

Let's get started with your hello world machine learning project in Python.
    
    
\section{Machine Learning in Python: Step-By-Step Tutorial (start here)}

In this section, we are going to work through a small machine learning project end-to-end.
    
Here is an overview of what we are going to cover:

\begin{itemize}
  \item Installing the Python and SciPy platform.
  \item Loading the dataset.
  \item Summarizing the dataset.
  \item Visualizing the dataset.
  \item Evaluating some algorithms.
  \item Making some predictions.
\end{itemize}    

Take your time. Work through each step.
    
Try to type in the commands yourself or copy-and-paste the commands to speed things up.
    
If you have any questions at all, please leave a comment at the bottom of the post.

\section{Step 1: Downloading, Installing and Starting Python SciPy}

Get the Python and SciPy platform installed on your system if it is not already.

I do not want to cover this in great detail, because others already have. This is already pretty straightforward, especially if you are a developer. If you do need help, ask a question in the comments.



\subsection{Install SciPy Libraries}

This tutorial assumes Python version 2.7 or 3.6+.

There are 5 key libraries that you will need to install. Below is a list of the Python SciPy libraries required for this tutorial:

\begin{itemize}
  \item scipy
  \item numpy
  \item matplotlib
  \item pandas
  \item sklearn
\end{itemize}

There are many ways to install these libraries. My best advice is to pick one method then be consistent in installing each library.

The \HREF{https://scipy.org/install/}{scipy installation page} provides excellent instructions for installing the above libraries on multiple different platforms, such as Linux, mac OS X and Windows. If you have any doubts or questions, refer to this guide, it has been followed by thousands of people.

\begin{itemize}
  \item On Mac OS X, you can use macports to install Python 3.6 and these libraries. For more information on macports, \HREF{https://www.macports.org/install.php}{see the homepage}.
  \item On Linux you can use your package manager, such as yum on Fedora to install RPMs.
\end{itemize}

If you are on Windows or you are not confident, I would recommend installing the free version of \HREF{https://www.anaconda.com/products/distribution}{Anaconda} that includes everything you need.

\medskip

Note: This tutorial assumes you have scikit-learn version 0.20 or higher installed.

\medskip

Need more help? See one of these tutorials:


\begin{itemize}
  \item \HREF{https://machinelearningmastery.com/setup-python-environment-machine-learning-deep-learning-anaconda/}{How to Setup a Python Environment for Machine Learning with Anaconda}
  \item \HREF{https://machinelearningmastery.com/linux-virtual-machine-machine-learning-development-python-3/}{How to Create a Linux Virtual Machine For Machine Learning With Python 3}
\end{itemize}

\subsection{Start Python and Check Versions}

It is a good idea to make sure your Python environment was installed successfully and is working as expected.

The script below will help you test out your environment. It imports each library required in this tutorial and prints the version.

Open a command line and start the python interpreter:

\medskip

\SHELL{python}
\medskip

I recommend working directly in the interpreter or writing your scripts and running them on the command line rather than big editors and IDEs. Keep things simple and focus on the machine learning not the toolchain.

Type or copy and paste the  script \ref{HelloWorldIrisCheck}:


\begin{code}
  \lstinputlisting[language=Python]{../Code/General/HelloWorldIris/HelloWorldIrisCheck.py}    
    
    \caption{Example ``Hello World Iris'' - Versionsprüfung der Packages}\label{HelloWorldIrisCheck}
\end{code}      


Here is the output I get on my OS X workstation:

\begin{lstlisting}
Python: 3.6.11 (default, Jun 29 2020, 13:22:26) 
[GCC 4.2.1 Compatible Apple LLVM 9.1.0 (clang-902.0.39.2)]
scipy: 1.5.2
numpy: 1.19.1
matplotlib: 3.3.0
pandas: 1.1.0
sklearn: 0.23.2
\end{lstlisting}

\medskip

Compare the above output to your versions.

Ideally, your versions should match or be more recent. The APIs do not change quickly, so do not be too concerned if you are a few versions behind, Everything in this tutorial will very likely still work for you.

If you get an error, stop. Now is the time to fix it.

If you cannot run the above script cleanly you will not be able to complete this tutorial.

My best advice is to Google search for your error message or post a question on \HREF{https://stackoverflow.com/questions/tagged/python}{Stack Exchange}.    
     
     
     
\section{Step 2: Load The Data}

We are going to use the iris flowers dataset. This dataset is famous because it is used as the “hello world” dataset in machine learning and statistics by pretty much everyone.
    
The dataset contains 150 observations of iris flowers. There are four columns of measurements of the flowers in centimeters. The fifth column is the species of the flower observed. All observed flowers belong to one of three species.
    
You can learn more about \HREF{https://en.wikipedia.org/wiki/Iris_flower_data_set}{this dataset on Wikipedia}.
    
In this step we are going to load the iris data from CSV file URL.
    
\subsection{Import libraries}

First, let's import all of the modules, functions and objects we are going to use in this tutorial, siehe \ref{HelloWorldIrisLoad}.

\begin{code}
    \lstinputlisting[language=Python,firstline=1, lastline=16]{../Code/General/HelloWorldIris/HelloWorldIrisLoad.py}    
    
    \caption{Example ``Hello World Iris'' - Load the Packages}\label{HelloWorldIrisLoad}
\end{code}      

    
Everything should load without error. If you have an error, stop. You need a working SciPy environment before continuing. See the advice above about setting up your environment.
    
    
\subsection{Load Dataset}

We can load the data directly from the UCI Machine Learning repository.
    
We are using pandas to load the data. We will also use pandas next to explore the data both with descriptive statistics and data visualization.

Note that we are specifying the names of each column when loading the data. This will help later when we explore the data, siehe \ref{HelloWorldIrisLoad}.



\begin{code}
    \lstinputlisting[language=Python,firstline=18, lastline=21]{../Code/General/HelloWorldIris/HelloWorldIrisLoad.py}    
    
    \caption{Example ``Hello World Iris'' - Load the Dataset}\label{HelloWorldIrisLoad2}
\end{code}      
 
The dataset should load without incident.
    
If you do have network problems, you can download the \href{../Code/General/HelloWorldIris/Iris.txt}{\FILE{iris.csv}} file into your working directory and load it using the same method, changing URL to the local file name.
 
%%%%%%%%%%%%%%%%%%%%%%%%%%%%%%%%%%%%%%%%
     
\section{Step 3: Summarize the Dataset}

Now it is time to take a look at the data.
    
In this step we are going to take a look at the data a few different ways:

\begin{enumerate}
  \item Dimensions of the dataset.
  \item Peek at the data itself.
  \item Statistical summary of all attributes.
  \item Breakdown of the data by the class variable.
\end{enumerate}    

Don't worry, each look at the data is one command. These are useful commands that you can use again and again on future projects.
    
\subsection{Dimensions of Dataset}

We can get a quick idea of how many instances (rows) and how many attributes (columns) the data contains with the shape property, see \ref{HelloWorldIrisLoad3}.
    
 
\begin{code}
    \lstinputlisting[language=Python,firstline=23, lastline=24]{../Code/General/HelloWorldIris/HelloWorldIrisLoad.py}    
    
    \caption{Example ``Hello World Iris'' - Shape of the Dataset} \label{HelloWorldIrisLoad3}
\end{code}      
 
You should see 150 instances and 5 attributes:

\begin{lstlisting}
(150, 5)
\end{lstlisting}
    
\subsection{Peek at the Data}

It is also always a good idea to actually eyeball your data, see \ref{HelloWorldIrisLoad4}.


\begin{code}
    \lstinputlisting[language=Python,firstline=26, lastline=27]{../Code/General/HelloWorldIris/HelloWorldIrisLoad.py}    
    
    \caption{Example ``Hello World Iris'' - Head of the Dataset} \label{HelloWorldIrisLoad4}
\end{code}      
 


You should see the first 20 rows of the data:

\begin{lstlisting}
    sepal-length  sepal-width  petal-length  petal-width        class
0            5.1          3.5           1.4          0.2  Iris-setosa
1            4.9          3.0           1.4          0.2  Iris-setosa
2            4.7          3.2           1.3          0.2  Iris-setosa
3            4.6          3.1           1.5          0.2  Iris-setosa
4            5.0          3.6           1.4          0.2  Iris-setosa
5            5.4          3.9           1.7          0.4  Iris-setosa
6            4.6          3.4           1.4          0.3  Iris-setosa
7            5.0          3.4           1.5          0.2  Iris-setosa
8            4.4          2.9           1.4          0.2  Iris-setosa
9            4.9          3.1           1.5          0.1  Iris-setosa
10           5.4          3.7           1.5          0.2  Iris-setosa
11           4.8          3.4           1.6          0.2  Iris-setosa
12           4.8          3.0           1.4          0.1  Iris-setosa
13           4.3          3.0           1.1          0.1  Iris-setosa
14           5.8          4.0           1.2          0.2  Iris-setosa
15           5.7          4.4           1.5          0.4  Iris-setosa
16           5.4          3.9           1.3          0.4  Iris-setosa
17           5.1          3.5           1.4          0.3  Iris-setosa
18           5.7          3.8           1.7          0.3  Iris-setosa
19           5.1          3.8           1.5          0.3  Iris-setosa
\end{lstlisting}
    
\subsection{Statistical Summary}

Now we can take a look at a summary of each attribute.
    
This includes the count, mean, the min and max values as well as some percentiles, .

\begin{code}
    \lstinputlisting[language=Python,firstline=29, lastline=30]{../Code/General/HelloWorldIris/HelloWorldIrisLoad.py}    
    
    \caption{Example ``Hello World Iris'' - Description of the Dataset} \label{HelloWorldIrisLoad4}
\end{code}      
    

 We can see that all of the numerical values have the same scale (centimeters) and similar ranges between 0 and 8 centimeters.
    
\begin{lstlisting}
       sepal-length  sepal-width  petal-length  petal-width
count    150.000000   150.000000    150.000000   150.000000
mean       5.843333     3.054000      3.758667     1.198667
std        0.828066     0.433594      1.764420     0.763161
min        4.300000     2.000000      1.000000     0.100000
25%        5.100000     2.800000      1.600000     0.300000
50%        5.800000     3.000000      4.350000     1.300000
75%        6.400000     3.300000      5.100000     1.800000
max        7.900000     4.400000      6.900000     2.500000
\end{lstlisting}
    
\subsection{Class Distribution}

Let's now take a look at the number of instances (rows) that belong to each class. We can view this as an absolute count.
    
\begin{code}
  \lstinputlisting[language=Python,firstline=29, lastline=30]{../Code/General/HelloWorldIris/HelloWorldIrisLoad.py}    
    
  \caption{Example ``Hello World Iris'' - Distribution of the Dataset} \label{HelloWorldIrisLoad5}
\end{code}      

We can see in \ref{HelloWorldIrisLoad5} that each class has the same number of instances (50 or 33\% of the dataset).

\begin{lstlisting}
class
Iris-setosa        50
Iris-versicolor    50
Iris-virginica     50
\end{lstlisting}    


\subsection{Complete Example}

For reference, we can tie all of the previous elements together into a single script.
    
The complete example is listed in  \ref{HelloWorldIrisSummarize}.

\begin{code}
  \lstinputlisting[language=Python]{../Code/General/HelloWorldIris/HelloWorldIrisLoad.py}    
    
  \caption{Example ``Hello World Iris'' - Complete Loading and Description of the Dataset} \label{HelloWorldIrisSummarize}
\end{code}      

    
%%%%%%%%%%%%%%%%%%%%%%%%%%%%%%%%%%%%%
\section{Step 4: Data Visualization}

We now have a basic idea about the data. We need to extend that with some visualizations.
    
We are going to look at two types of plots:

\begin{enumerate}
  \item Univariate plots to better understand each attribute.
  \item Multivariate plots to better understand the relationships between attributes.
\end{enumerate}    

\subsection{Univariate Plots}

We start with some univariate plots, that is, plots of each individual variable.
    
Given that the input variables are numeric, we can create box and whisker plots of each, see \ref{HelloWorldIrisVisualize}.


\begin{code}
  \lstinputlisting[language=Python,firstline=11,lastline=13]{../Code/General/HelloWorldIris/HelloWorldIrisVisualize.py}    
    
  \caption{Example ``Hello World Iris'' - Box and Whisker Plots} \label{HelloWorldIrisVisualize}
\end{code}      
    
This gives us a much clearer idea of the distribution of the input attributes. see \ref{HelloWorldIrisBoxPlots}.
  

\begin{figure}
  \centering
    
  \includegraphics[width=\textwidth]{images/HelloWorldIris/HelloWorldIrisBoxPlots}

  \caption{Box and Whisker Plots for Each Input Variable for the Iris Flowers Dataset}\label{HelloWorldIrisBoxPlots}
\end{figure}    
    
We can also create a histogram of each input variable to get an idea of the distribution, see \ref{HelloWorldIrisVisualize2}.
    
\begin{code}
  \lstinputlisting[language=Python,firstline=15,lastline=17]{../Code/General/HelloWorldIris/HelloWorldIrisVisualize.py}    
    
  \caption{Example ``Hello World Iris'' - Histograms} \label{HelloWorldIrisVisualize2}
\end{code}      

It looks like perhaps two of the input variables have a Gaussian distribution. This is useful to note as we can use algorithms that can exploit this assumption, see \ref{HelloWorldIrisHistogram}.
    
\begin{figure}
  \centering
    
  \includegraphics[width=\textwidth]{images/HelloWorldIris/HelloWorldIrisHistogram}
    
  \caption{Histogram Plots for Each Input Variable for the Iris Flowers Dataset}\label{HelloWorldIrisHistogram}
\end{figure}    
    
  
    
\subsection{Multivariate Plots}

Now we can look at the interactions between the variables.
    
First, let's look at scatterplots of all pairs of attributes. This can be helpful to spot structured relationships between input variables, see \ref{HelloWorldIrisVisualize3}.
    
    
\begin{code}
  \lstinputlisting[language=Python,firstline=19,lastline=21]{../Code/General/HelloWorldIris/HelloWorldIrisVisualize.py}    
    
  \caption{Example ``Hello World Iris'' - Scatter Plot Matrix} \label{HelloWorldIrisVisualize3}
\end{code}      
    
Note the diagonal grouping of some pairs of attributes in image \ref{HelloWorldIrisMatrixPlot}. This suggests a high correlation and a predictable relationship.
    
\begin{figure}
  \centering
    
  \includegraphics[width=\textwidth]{images/HelloWorldIris/HelloWorldIrisMatrixPlot}
    
  \caption{Scatter Matrix Plot for Each Input Variable for the Iris Flowers Dataset}\label{HelloWorldIrisMatrixPlot}
\end{figure}    

   
    
\subsection{Complete Example}

For reference, we can tie all of the previous elements together into a single script.
    
The complete example is listed in \ref{HelloWorldIrisVisualize4}.

\begin{code}
    \lstinputlisting[language=Python]{../Code/General/HelloWorldIris/HelloWorldIrisVisualize.py}    
    
    \caption{Example ``Hello World Iris'' - Scatter Plot Matrix} \label{HelloWorldIrisVisualize4}
\end{code}      
    
                
 
       
\section{Step 5: Evaluate Some Algorithms}

Now it is time to create some models of the data and estimate their accuracy on unseen data.
    
Here is what we are going to cover in this step:

\begin{enumerate}
  \item Separate out a validation dataset.
  \item Set-up the test harness to use 10-fold cross validation.
  \item Build multiple different models to predict species from flower measurements
  \item Select the best model.
\end{enumerate}

\subsection{Create a Validation Dataset}

We need to know that the model we created is good.
    
Later, we will use statistical methods to estimate the accuracy of the models that we create on unseen data. We also want a more concrete estimate of the accuracy of the best model on unseen data by evaluating it on actual unseen data.
    
That is, we are going to hold back some data that the algorithms will not get to see and we will use this data to get a second and independent idea of how accurate the best model might actually be.
    
We will split the loaded dataset into two, 80\% of which we will use to train, evaluate and select among our models, and 20\% that we will hold back as a validation dataset, see \ref{HelloWorldIrisCompare2}.
  

\begin{code}
    \lstinputlisting[language=Python,firstline=19,lastline=23]{../Code/General/HelloWorldIris/HelloWorldIrisCompare.py}    
    
    \caption{Example ``Hello World Iris'' - Plot Algorithm Comparison} \label{HelloWorldIrisCompare2}
\end{code}        

You now have training data in the \PYTHON{X\_train} and \PYTHON{Y\_train} for preparing models and a \PYTHON{X\_validation} and \PYTHON{Y\_validation} sets that we can use later.
    
Notice that we used a python slice to select the columns in the NumPy array. If this is new to you, you might want to check-out this post:


\begin{itemize}
  \item \HREF{https://machinelearningmastery.com/index-slice-reshape-numpy-arrays-machine-learning-python/}{How to Index, Slice and Reshape NumPy Arrays for Machine Learning in Python}
\end{itemize}     

\subsection{Test Harness}

We will use stratified 10-fold cross validation to estimate model accuracy.
    
This will split our dataset into 10 parts, train on 9 and test on 1 and repeat for all combinations of train-test splits.
    
Stratified means that each fold or split of the dataset will aim to have the same distribution of example by class as exist in the whole training dataset.
    
For more on the k-fold cross-validation technique, see the tutorial:
    
\begin{itemize}
    \item \HREF{https://machinelearningmastery.com/k-fold-cross-validation/}{A Gentle Introduction to k-fold Cross-Validation}
\end{itemize}     
    
We set the random seed via the argument \PYTHON{random\_state} to a fixed number to ensure that each algorithm is evaluated on the same splits of the training dataset.
    
The specific random seed does not matter, learn more about pseudorandom number generators here:
    
\begin{itemize}
  \item \HREF{https://machinelearningmastery.com/introduction-to-random-number-generators-for-machine-learning/}{Introduction to Random Number Generators for Machine Learning in Python}
\end{itemize}     
    
We are using the metric of ``accuracy'' to evaluate models.
    
This is a ratio of the number of correctly predicted instances divided by the total number of instances in the dataset multiplied by 100 to give a percentage (e.g. 95\% accurate). We will be using the scoring variable when we run build and evaluate each model next.
    
    
\subsection{Build Models}

We don't know which algorithms would be good on this problem or what configurations to use.
    
We get an idea from the plots that some of the classes are partially linearly separable in some dimensions, so we are expecting generally good results.
    
Let's test 6 different algorithms:

\begin{itemize}
  \item Logistic Regression (LR)
  \item Linear Discriminant Analysis (LDA)
  \item k-Nearest Neighbors (KNN).
  \item Classification and Regression Trees (CART).
  \item Gaussian Naive Bayes (NB).
  \item Support Vector Machines (SVM).
\end{itemize}

This is a good mixture of simple linear (LR and LDA), nonlinear (KNN, CART, NB and SVM) algorithms.
    
Let's build and evaluate our models, see \ref{HelloWorldIrisCompare3}.

\begin{code}
    \lstinputlisting[language=Python,firstline=25,lastline=42]{../Code/General/HelloWorldIris/HelloWorldIrisCompare.py}    
    
    \caption{Example ``Hello World Iris'' - Using the Algorithms} \label{HelloWorldIrisCompare3}
\end{code}        
    
    
\subsection{Select Best Model}

We now have 6 models and accuracy estimations for each. We need to compare the models to each other and select the most accurate.
    
Running the example above, we get the following raw results:

\begin{lstlisting}
LR: 0.960897 (0.052113)
LDA: 0.973974 (0.040110)
KNN: 0.957191 (0.043263)
CART: 0.957191 (0.043263)
NB: 0.948858 (0.056322)
SVM: 0.983974 (0.032083)
\end{lstlisting}    

\medskip

Note: Your  \HREF{https://machinelearningmastery.com/different-results-each-time-in-machine-learning/}{results may vary} given the stochastic nature of the algorithm or evaluation procedure, or differences in numerical precision. Consider running the example a few times and compare the average outcome.
  
\medskip

In this case, we can see that it looks like Support Vector Machines (SVM) has the largest estimated accuracy score at about 0.98 or 98\%.
    
We can also create a plot of the model evaluation results and compare the spread and the mean accuracy of each model. There is a population of accuracy measures for each algorithm because each algorithm was evaluated 10 times (via 10 fold-cross validation).
    
A useful way to compare the samples of results for each algorithm is to create a box and whisker plot for each distribution and compare the distributions, see \ref{HelloWorldIrisCompare1}.

\begin{code}
  \lstinputlisting[language=Python,firstline=44,lastline=47]{../Code/General/HelloWorldIris/HelloWorldIrisCompare.py}    
    
  \caption{Example ``Hello World Iris'' - Plot Algorithm Comparison} \label{HelloWorldIrisCompare1}
\end{code}    

We can see in image \ref{HelloWorldIrisComparePlot} that the box and whisker plots are squashed at the top of the range, with many evaluations achieving 100\% accuracy, and some pushing down into the high 80\% accuracies.
    
\begin{figure}
    \centering
    
    \includegraphics[width=\textwidth]{images/HelloWorldIris/HelloWorldIrisCompare}
    
    \caption{Box and Whisker Plot Comparing Machine Learning Algorithms on the Iris Flowers Dataset}\label{HelloWorldIrisComparePlot}
\end{figure}    

    
\subsection{Complete Example}

For reference, we can tie all of the previous elements together into a single script.
    
The complete example is listed in \ref{HelloWorldIrisCompare}.

\begin{code}
  \lstinputlisting[language=Python]{../Code/General/HelloWorldIris/HelloWorldIrisCompare.py}    
    
  \caption{Example ``Hello World Iris'' - Scatter Plot Matrix} \label{HelloWorldIrisCompare}
\end{code}      
   
                

\section{Step 6: Make Predictions}

We must choose an algorithm to use to make predictions.
    
The results in the previous section suggest that the SVM was perhaps the most accurate model. We will use this model as our final model.
    
Now we want to get an idea of the accuracy of the model on our validation set.
    
This will give us an independent final check on the accuracy of the best model. It is valuable to keep a validation set just in case you made a slip during training, such as overfitting to the training set or a data leak. Both of these issues will result in an overly optimistic result.
    
    
\subsection{Make Predictions}

We can fit the model on the entire training dataset and make predictions on the validation dataset, see \ref{HelloWorldIrisPredition}.

\begin{code}
    \lstinputlisting[language=Python, firstline=20, lastline=23]{../Code/General/HelloWorldIris/HelloWorldIrisPreditcion.py}    
    
    \caption{Example ``Hello World Iris'' - Make Predictions}\label{HelloWorldIrisPredition}
\end{code}      
    

You might also like to make predictions for single rows of data. For examples on how to do that, see the tutorial:

\medskip
    
\HREF{https://machinelearningmastery.com/make-predictions-scikit-learn/}{How to Make Predictions with scikit-learn}

\medskip

You might also like to save the model to file and load it later to make predictions on new data. For examples on how to do this, see the tutorial:
    
\medskip

\HREF{https://machinelearningmastery.com/save-load-machine-learning-models-python-scikit-learn/}{Save and Load Machine Learning Models in Python with scikit-learn}
    
\subsection{Evaluate Predictions}

We can evaluate the predictions by comparing them to the expected results in the validation set, then calculate classification accuracy, as well as a confusion matrix and a classification report.
    
\begin{code}
    \lstinputlisting[language=Python, firstline=25, lastline=28]{../Code/General/HelloWorldIris/HelloWorldIrisPreditcion.py}    
    
    \caption{Example ``Hello World Iris'' - Evaluation}\label{HelloWorldIrisEvaluation}
\end{code}      
    
We can see that the accuracy is 0.966 or about 96\% on the hold out dataset, see \ref{HelloWorldIrisEvaluation}.
    
The confusion matrix provides an indication of the errors made.
    
Finally, the classification report provides a breakdown of each class by precision, recall, f1-score and support showing excellent results (granted the validation dataset was small).
   
\medskip

\begin{lstlisting}
0.9666666666666667
[[11  0  0]
 [ 0 12  1]
 [ 0  0  6]]
                  precision    recall  f1-score   support
    
    Iris-setosa       1.00      1.00      1.00        11
Iris-versicolor       1.00      0.92      0.96        13
 Iris-virginica       0.86      1.00      0.92         6
    
       accuracy                           0.97        30
      macro avg       0.95      0.97      0.96        30
   weighted avg       0.97      0.97      0.97        30
\end{lstlisting}    
    
\subsection{Complete Example}

For reference, we can tie all of the previous elements together into a single script.

The complete example is listed \ref{HelloWorldIrisComplete}.

\medskip

\begin{code}
  \lstinputlisting[language=Python]{../Code/General/HelloWorldIris/HelloWorldIrisPreditcion.py}    
    
  \caption{Example ``Hello World Iris'' -- Complete}\label{HelloWorldIrisComplete}
\end{code}      
        
\section{You Can Do Machine Learning in Python}
 
Work through the tutorial above. It will take you 5-to-10 minutes, max!
     
You do not need to understand everything. (at least not right now) Your goal is to run through the tutorial end-to-end and get a result. You do not need to understand everything on the first pass. List down your questions as you go. Make heavy use of the \PYTHON{help(``FunctionName'')} help syntax in Python to learn about all of the functions that you're using.
     
You do not need to know how the algorithms work. It is important to know about the limitations and how to configure machine learning algorithms. But learning about algorithms can come later. You need to build up this algorithm knowledge slowly over a long period of time. Today, start off by getting comfortable with the platform.
     
You do not need to be a Python programmer. The syntax of the Python language can be intuitive if you are new to it. Just like other languages, focus on function calls (e.g. \PYTHON{function()}) and assignments (e.g. \PYTHON{a = ``b''}). This will get you most of the way. You are a developer, you know how to pick up the basics of a language real fast. Just get started and dive into the details later.
     
You do not need to be a machine learning expert. You can learn about the benefits and limitations of various algorithms later, and there are plenty of posts that you can read later to brush up on the steps of a machine learning project and the importance of evaluating accuracy using cross validation.
     
What about other steps in a machine learning project. We did not cover all of the steps in a machine learning project because this is your first project and we need to focus on the key steps. Namely, loading data, looking at the data, evaluating some algorithms and making some predictions. In later tutorials we can look at other data preparation and result improvement tasks.
     
\section{Summary}

In this post, you discovered step-by-step how to complete your first machine learning project in Python.
    
You discovered that completing a small end-to-end project from loading the data to making predictions is the best way to get familiar with a new platform.
    
\section{Your Next Step}

Do you work through the tutorial?

\begin{enumerate}
  \item Work through the above tutorial.
  \item List any questions you have.
  \item Search-for or research the answers.
  \item Remember, you can use the \PYTHON{help(``FunctionName'')} in Python to get help on any function.
\end{enumerate}
  
Do you have a question?

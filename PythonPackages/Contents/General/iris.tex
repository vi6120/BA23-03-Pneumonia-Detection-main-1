%%%%%%
%
% $Autor: Wings $
% $Datum: 2021-05-14 $
% $Pfad: GitLab/Bilderkennung/Projects $
% $Dateiname: iris
% $Version: 4620 $
%
%%%%%%

%todo Manche Texte finden sich in TensorFlow.tex wieder. Dies ist zu ändern.
%Quelle: https://riptutorial.com/de/scikit-learn/example/6801/beispieldatensatze


\subsection{Datensatz Fisher's Iris Data Set}

Die Erkennung der Iris ist ein weiterer Klassiker zur Bildklassifizierung. Auch zu diesem Klassiker gibt es mehrere Tutorials. Der Datensatz Fisher's Iris Data Set\index{Datensatz!Fisher's Iris Data Set} wurde in R.A.~Fishers klassischer Arbeit von 1936 verwendet und kann im UCI Machine Learning Repository \index{UCI Machine Learning Repository} gefunden werden. \cite{Fisher:1936,Iris:2021,Schutten:2016} Es wurde vier Merkmale der Blumen Iris Setosa, Iris Versicolour und Iris Virginica vermessen. Für jede der drei Klassen stehen 50 Datensätze mit vier Attributen zur Verfügung. Gemessen wurden dabei jeweils die Breite und die Länge des Kelchblatts (Sepalum) sowie des Kronblatts (Petalum) in Zentimeter. \cite{Anderson:1935,Sahni:2006}. Eine Blumenart ist linear von den anderen beiden trennbar, aber die anderen beiden sind nicht linear voneinander trennbar.


Auf den Datensatz kann an mehreren Stellen zugriffen werden:

\begin{itemize}
    \item \href{https://archive.ics.uci.edu/ml/datasets/iris}{Archiv von Datensätzen für das Maschinelle Lernen der Univerität von Kalifornieren in Irvine} \cite{UCIIris:2021}
    \item \href{https://scikit-learn.org/stable/auto_examples/datasets/plot_iris_dataset.html}{Python Paket \FILE{skikitlearn}} \cite{scikit-learn:2011}
    \item \href{https://www.kaggle.com/uciml/iris}{Kaggle - Website für Maschinelles Lernen} \cite{KaggleIris:2018}
    %\item \url{https://gist.github.com/curran/a08a1080b88344b0c8a7}
\end{itemize}

Den Datensatz steht auf verschiedenen Webseiten zur Verfügung, allerdings muss auf den Aufbau des Datensatzes geachtet werden. Da es sich um eine Datei im Format CSV handelt, ist der Aufbau spaltenorientiert. In der Regel ist in der ersten Zeile der Titel der einzelnen Spalte angegeben: \texttt{sepal\_length}, \texttt{sepal\_width}, \texttt{petal\_length}, \texttt{petal\_width} und \texttt{species}. Die Werte sind in Zentimeter angegeben, die Spezie ist mit \texttt{setosa} für Iris Setosa, \texttt{versicolor} für Iris Versicolor und \texttt{virginica} für die Spezie Iris Virginica angegeben.
Die Spalten in diesem Datensatz sind:

\begin{itemize}
  \item Laufende Nummer
  \item Länge des  Kelchblatts (Sepal) in Zentimeter
  \item Breite des  Kelchblatts (Sepal) in Zentimeter
  \item Länge des Blütenblatts (Petal) in Zentimeter
  \item Breite des Blütenblatts (Petal) in Zentimeter
  \item Art    
\end{itemize}




Das Ziel ist die Klassifizierung der drei verschiedenen Iris-Arten anhand der Länge und Breite von Kelchblatt und Blütenblatt. Da der Datensatz mit der Bibliothek \PYTHON{sklearn} ausgeliefert wird, wird dieser Zugang gewählt. 

\begin{code}
  \begin{lstlisting}[language=MyPython, numbers=left,label={src:irisimport}]
from sklearn.datasets import load_iris
iris = load_iris()
  \end{lstlisting}
  \caption{Laden des Datensatzes Fisher's Iris Data Set\index{Datensatz!Fisher's Iris Data Set}}
\end{code}


Der Datensatz ist ein \PYTHON{dictionary}. Seine Schlüssel kann man sich leicht anzeigen lassen:

\begin{lstlisting}[language=MyPython, numbers=left]
>>> iris.keys()
\end{lstlisting}

Die zugehörige Ausgabe ist wie folgt:

\begin{lstlisting}[numbers=none]
    dict\_keys(['data', 'target', 'frame', 'target\_names', 'DESCR', 'feature\_names', 'filename'])
\end{lstlisting}



Die einzelnen Elemente kann man sich nun ansehen. Nach Eingabe des Befehls 

\begin{lstlisting}[language=MyPython, numbers=left]
iris['DESCR']
\end{lstlisting}

wird eine ausführliche Beschreibung ausgegeben:

\begin{code}
\begin{lstlisting}[language=MyPython, numbers=left]
'.. _iris_dataset:
    
Iris plants dataset
--------------------
    
**Data Set Characteristics:**
    
:Number of Instances: 150 (50 in each of three classes)
:Number of Attributes: 4 numeric, predictive attributes and the class
:Attribute Information:
   - sepal length in cm
   - sepal width in cm
   - petal length in cm
   - petal width in cm
   - class:
       - Iris-Setosa
       - Iris-Versicolour
       - Iris-Virginica
      
   :Summary Statistics:
   
   ============== ==== ==== ======= ===== ====================
   Min  Max   Mean    SD   Class Correlation
   ============== ==== ==== ======= ===== ====================
   epal length:    4.3  7.9   5.84   0.83    0.7826
   sepal width:    2.0  4.4   3.05   0.43   -0.4194
   petal length:   1.0  6.9   3.76   1.76    0.9490  (high!)
   petal width:    0.1  2.5   1.20   0.76    0.9565  (high!)
   ============== ==== ==== ======= ===== ====================
   
   :Missing Attribute Values: None
   :Class Distribution: 33.3% for each of 3 classes.
   :Creator: R.A. Fisher
   :Donor: Michael Marshall (MARSHALL%PLU@io.arc.nasa.gov)
   :Date: July, 1988
   
   The famous Iris database, first used by Sir R.A. Fisher. The dataset is taken
   from Fisher\'s paper. Note that it\'s the same as in R, but not as in the UCI\nMachine Learning Repository, which has two wrong data points.
   
   This is perhaps the best known database to be found in the
   pattern recognition literature.  Fisher\'s paper is a classic in the field and
   is referenced frequently to this day.  (See Duda & Hart, for example.)  The
   data set contains 3 classes of 50 instances each, where each class refers to a\ntype of iris plant.  One class is linearly separable from the other 2; the\nlatter are NOT linearly separable from each other.
   
   .. topic:: References
   
   - Fisher, R.A. "The use of multiple measurements in taxonomic problems"
     Annual Eugenics, 7, Part II, 179-188 (1936); also in "Contributions to
     Mathematical Statistics" (John Wiley, NY, 1950).
   - Duda, R.O., & Hart, P.E. (1973) Pattern Classification and Scene Analysis.
     (Q327.D83) John Wiley & Sons.  ISBN 0-471-22361-1.  See page 218.
   - Dasarathy, B.V. (1980) "Nosing Around the Neighborhood: A New System
     Structure and Classification Rule for Recognition in Partially Exposed
     Environments".  IEEE Transactions on Pattern Analysis and Machine
     Intelligence, Vol. PAMI-2, No. 1, 67-71.
   - Gates, G.W. (1972) "The Reduced Nearest Neighbor Rule".  IEEE Transactions
     on Information Theory, May 1972, 431-433.
   - See also: 1988 MLC Proceedings, 54-64.  Cheeseman et al"s AUTOCLASS II
     conceptual clustering system finds 3 classes in the data.
   - Many, many more ...'
\end{lstlisting}
\caption{Beschreibung des Datensatzes Fisher's Iris Data Set\index{Datensatz!Fisher's Iris Data Set}}
\end{code}

Mit der Eingabe

\begin{lstlisting}[language=MyPython, numbers=left]
    iris['feature_names']
\end{lstlisting}

erhält man die Namen der Attribute:

\begin{lstlisting}[numbers=none]
    ['sepal length (cm)',
    'sepal width (cm)',
    'petal length (cm)',
    'petal width (cm)']
\end{lstlisting}

Die Namen der Blumen, die nach der Eingabe

\begin{lstlisting}[language=MyPython, numbers=left]
    iris['target_names']
\end{lstlisting}

angezeigt werden, sind

\begin{lstlisting}[numbers=none]
    array(['setosa', 'versicolor', 'virginica'], dtype='<U10')
\end{lstlisting}

Zur weiteren Untersuchung werden die Daten mit den Überschriften in einen Datenrahmen aufgenommen.

\begin{lstlisting}[language=MyPython, numbers=left]
    X = pd.DataFrame(data = iris.data, columns = iris.feature_names)
    print(X.head())
\end{lstlisting}

Der Befehl \PYTHON{(X.head())} zeigt -- wie in Abbildung~\ref{TensorFlowHead} -- den Kopf des Datenrahmens an. Ersichtlich ist, dass jeder Datensatz aus vier Werten besteht. 

\begin{figure}[H]
    \GRAPHICSC{0.6}{1.0}{TensorFlow/IrisHead}
    \caption{Kopfzeilen des Datensatzes Fisher's Iris Data Set\index{Datensatz!Fisher's Iris Data Set}}\label{TensorFlowHead}
\end{figure}

Jeder Datensatz enthält auch schon in dem Schlüssel \PYTHON{target} seine Klassifikation. In der Abbildung~\ref{TensorFlowHeadType} ist dies für die ersten Datensätze aufgeführt.

\begin{lstlisting}[language=MyPython, numbers=left]
    y = pd.DataFrame(data=iris.target, columns = ['irisType'])
    y.head()
\end{lstlisting}

\begin{figure}[H]
    \GRAPHICSC{1.0}{1.0}{TensorFlow/IrisHeadType}
    \caption{Ausgabe der Kategorien des Datensatzes Fisher's Iris Data Set\index{Datensatz!Fisher's Iris Data Set}}\label{TensorFlowHeadType}
\end{figure}

Mittels des Befehls 

\begin{lstlisting}[language=MyPython, numbers=left]
    y.irisType.value_counts()
\end{lstlisting}

kann ermittel werden, wie viele Klassen vorliegen. Die Ausgabe des Befehls wird in der Abbildung~\ref{TensorFlowIrisTypes}
gezeigt; es ergeben sich 3 Klassen mit den Nummern $0$, $1$ und $2$. Jeweils 50 Datensätze sind ihnen zugeordnet.

\begin{figure}[H]
    \GRAPHICSC{1.0}{1.0}{TensorFlow/IrisTypes}
    \caption{Namen der Kategorien im Datensatz Fisher's Iris Data Set\index{Datensatz!Fisher's Iris Data Set}}\label{TensorFlowIrisTypes}
\end{figure}

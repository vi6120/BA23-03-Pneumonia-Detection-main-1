%%%
%
% $Autor: Wings $
% $Datum: 2021-05-14 $
% $Dateiname: NumPy.tex
% $Version: 4620 $
%
% !TeX spellcheck = GB
% !TeX program = pdflatex
% !TeX encoding = utf8

%%%

\chapter{Description of the Python Package: NumPy}

Numerical Python (NumPy) is one of the most fundamental packages available in python for numerical computation. It is a general-purpose array-processing package which provides high-performance multidimensional arrays and appropriate tools to work with them. NumPy is capable of storing generic multi-dimensional data. In NumPy, dimensions are known as axes and the rank denotes the number of axes present. NumPy's array class is termed as ndarray. 


\section{Functions:}

\begin{itemize}
    \item Computes basic array operations such as addition, multiplication, slicing, flatten, reshape and array indexing.
    \item Computes advanced array operations such as stacking arrays, splitting into sections and broadcasting arrays
    \item NumPy works with either Date Time or Linear Algebra
\end{itemize}



\section{Features:}

\begin{itemize}
    \item Provides pre-compiled functions for numerical routines 
    \item Array-oriented computing for better efficiency
    \item NumPy supports an object-oriented approach
    \item It is compact and performs faster computations with vectorization
\end{itemize}



\section{Applications:}

\begin{itemize}
    \item Predominantly used in data-analysis applications.
    \item Used for creating powerful N-dimensional array
    \item Forms the base of other libraries, such as SciPy and scikit-learn
    \item Used as a replacement of MATLAB when used with SciPy and matplotlib
\end{itemize}


\section{Installation}

To install Numpy, you can use the pip package manager by running the command \SHELL{pip install numpy} in your terminal.




\section{Further Reading:}

\begin{itemize}
    \item The Numpy documentation (\url{https://numpy.org/doc/}) provides a comprehensive overview of the library's features and usage.
    \item The Scipy website (\url{https://scipy.org/}) also provides additional resources for scientific computing with Python, including tutorials and a user guide.
\end{itemize}


\begin{code}
    \lstinputlisting[language=Python]{../Code/General/NumPy/NumPySample.py}
    
    \caption{Simple example for NumPy}
\end{code}   


%%%%%%%%
%
% $Autor: Vikas Ramaswamy$
% $Date: 2023-04-09$
% $Homework: Updated literature$
% $Project: Pnuemonia Detection$
% $Version: 1.0.0 $
%
%
% !TeX encoding = utf8
% bibtex
%
%%%%%%




\Mysection{Pneumonia Detection Using CNN based Feature Extraction}
\STANDARD{Pneumonia Detection Using CNN based Feature Extraction}  
{
	
	Keywords: DensetNet, Deep Convolutional Neural Networks, SVM, Transfer Learning, Random Forest, Naive Bayes, K-nearest neighbors, Feature extraction.\newline
	
	A deep learning-based approach for diagnosing pneumonia from chest X-ray pictures is presented in the paper "Pneumonia Detection Using CNN based Feature Extraction," which was published at the 2019 IEEE International Conference on Electrical, Computer, and Communication Technologies. The authors extracted information from the photos using a Convolutional Neural Network (CNN) model and then classified them as normal or pneumonia afflicted. \newpage
	
	The paper describes in detail the dataset used, the architecture of the CNN model, and the evaluation measures used to assess the model's performance. The results reveal that the suggested method detects pneumonia with high accuracy from chest X-ray pictures. Overall, this paper proposes a viable method for early identification of pneumonia, which can aid in both the detection and treatment of this potentially deadly disease.\cite{Varshni:2019}
	
	
}


\Mysection{Pneumonia Detection Using Convolutional Neural Networks (CNNs)}
\STANDARD{Pneumonia Detection Using Convolutional Neural Networks (CNNs)}
{
	Keywords: Kaggle, Keras, ReLU, Max-pooling, Forward and backward propagation, Overfitting, Pooling layer.\newline
	
	The paper titled "Pneumonia Detection Using Convolutional Neural Networks (CNNs)" by Kaushik et al. presents a deep learning-based approach for detecting pneumonia from chest X-ray images.  The authors classified the images as normal or pneumonia-affected using characteristics extracted from the photos using a CNN algorithm. The dataset employed, the CNN model's architecture, and the evaluation measures used to gauge the model's performance are all thoroughly described in the study.\newpage
	
	The outcomes demonstrate that the suggested method successfully detects pneumonia from chest X-ray pictures with high accuracy. By outlining a potential strategy for the early diagnosis of pneumonia, the authors have significantly advanced the field of medical picture analysis.\cite{Kaushik:2020}
	
}

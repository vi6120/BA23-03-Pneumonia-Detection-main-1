%%%
%
% $Autor: Vikas Ramaswamy$
% $Datum: 2023-03-17 11:15:45Z $
% $Pfad: GitLab/CornerBlending $
% $Dateiname: FirstChapter
% $Version: 4620 $
%
% !TeX spellcheck = de_DE/GB
%
%%%


\chapter{Open Questions}

In the context of pneumonia detection using ML models, there are several open questions and future directions for research that can be explored. Some of these include:

\begin{enumerate}
	\item Can the model be further improved by using more advanced transfer learning techniques?
	\item Can the model be trained on additional data sources to improve its accuracy and reliability?
	\item Can the model be adapted for use in real-time pneumonia detection applications?
	\item Can the model be modified to detect other respiratory conditions, such as tuberculosis or COVID-19?
	\item How can the model be made more interpretable for medical professionals to better understand its predictions?
	\item Can we develop a real-time pneumonia detection application that can provide accurate and timely diagnoses to patients and medical professionals?
	\item  Can the model be Integrated into electronic medical records (EMRs) to provide seamless and efficient pneumonia diagnosis and treatment recommendations?
\end{enumerate}

\chapter{Conclusion}

\section{Conclusion}

In terms of accuracy, our proposed pneumonia detection system, which incorporates supervised ML algorithms, outperforms previously existing methods. This enhancement was obtained by implementing a convolutional neural network (CNN) trained on a huge dataset of X-ray images.\\

We can investigate the usage of advanced image processing techniques such as texture analysis, form analysis, and edge detection to further increase the accuracy of our system. Additionally, uploading X-ray images in a specific orientation can help standardize the dataset and improve overall system efficiency.\\

To boost accuracy, we can apply multi-class classification algorithms to the dataset, such as SVM, Random Forest, and K-Nearest Neighbor. These algorithms have been demonstrated to be effective in image categorization tasks.\\

Finally, it is crucial to mention that our application is created with the purpose of safeguarding the best interests of patients and healthcare professionals in mind. We believe that our technology can assist healthcare providers in swiftly and accurately diagnosing pneumonia, resulting in better patient outcomes.\\

\section{Next Step}

The potential future work that can be done is mentioned below\\


\begin{itemize}
\item	Artificial intelligence will make it possible to progressively move towards automating repetitive and time-consuming tasks, such as screening for lung nodules and identifying image-based biomarkers. These developments will optimistically allow disease characterization in a non-invasive and repeatable way, improving therapeutic management in order to achieve the goal of personalized medicine \cite{cellina2022artificial}



\item In the future, we may investigate techniques such as contrast enhancement of the images or other pre-processing steps to improve the image quality. We may also consider using segmentation of the lung image before classification to enable the CNN models to achieve improved feature extraction \cite{kundu2021pneumonia}



\item While X-ray imaging is widely used for pneumonia detection, other imaging modalities such as CT scans and ultrasound can also be used. Future work can focus on developing multi-modal approaches that combine information from different imaging modalities for more accurate and reliable diagnosis \\

\item Real-world Deployment:- Preparing  the pneumonia detection system for real-world deployment by considering deployment environments, scalability, security and regulatory compliance.\\

\item Validatoin:- Considering factors such as sensitivity, specificity, and impact on patient outcomes we need to conduct clinical trials and validation studies to assess the performance and clinical utility of the pneumonia detection.\\


\item Monitoring and Model Updates:- TO ensure ongoing accuracy and effectiveness the system need to be monitor and update the model on regular basis\\

\item Real-time Detection: Investigate techniques to enable real-time pneumonia detection, allowing for immediate decision-making and intervention. This may involve optimizing the model architecture and employing efficient inference techniques to achieve low-latency performance.\\

\item  Deployment in Resource-Constrained Settings: Focus on adapting the pneumonia detection system for resource-constrained environments, such as low-resource healthcare settings or areas with limited access to medical facilities. This may involve optimizing the model's computational requirements, developing lightweight versions of the algorithm, or exploring edge computing solutions\\

\end{itemize}

\section{To Do}

\begin{itemize}

\item For creating a more productive ML model, it is required to have a larger volume and variety of datasets to train the model. The lab-based datasets are limited to be used for effective training of ML models in a real-time scenario like in hospitals or medical institutions. Therefore, we need to have a solution of using real-time data to fulfill the requirements of having a more significant variety of data.\cite{kareem2022review}
	
\item Clinical and demographic data, such as patient age, medical history, and symptoms, can provide important context for pneumonia diagnosis. Future work can focus on integrating this information into the image-based deep learning models to improve their accuracy. \cite{han2021pneumonia}

\item Data collection:-  To enhance the performance and generalization we need to gather a larger and more diverse dataset for training and validation.\\

\item Data validation and Evaluation:- Collaborate with healthcare professionals to validate the accuracy and clinical relevance of the pneumonia detection. This is help to refine the algorithm and address any potential limitations, ensuring that the system aligns with the requirements and standards of medical practice.\\

\item Improvement:- Improve the process of determining how certain or confident a pneumonia detection is about its predictions. Need to make sure how the system is about whether an image contains pneumonia or not. If the uncertainty is low, it means the system is very confident in its prediction. On the other hand, if the uncertainty is high, it indicates that the system is less certain about its prediction.\\

\item Transfer Learning from Multiple Domains: Instead of relying solely on a single pre-trained model, explore transfer learning from multiple domains or datasets. This can involve utilizing pre-trained models trained on different medical imaging datasets or even non-medical image datasets that exhibit similar visual patterns or structures.\\

\end{itemize}

\section{Unanswered points}

\begin{itemize}
	\item Real-world Variability: Account for real-world variability and challenges in pneumonia detection, such as imaging artifacts, variations in patient positioning, or image quality issues. Augment the dataset or develop robust algorithms that can handle these common challenges encountered in clinical practice.
	\item Multi-Modality Fusion: Explore approaches for integrating information from multiple imaging modalities, such as chest X-rays and CT scans, to improve the accuracy and reliability of pneumonia detection. Fusion techniques can leverage complementary information from different modalities and enhance the overall performance.
	\item Uncertainty Estimation: Develop techniques to estimate uncertainty in pneumonia detection models. Uncertainty estimation provides insights into the confidence or reliability of model predictions, enabling clinicians to make more informed decisions based on the model's output.
	\item External Validation and Generalization: Conduct external validation of the pneumonia detection model on diverse and independent datasets to assess its generalization capabilities. Evaluate the model's performance across different patient populations, geographic regions, or healthcare institutions to ensure its reliability and effectiveness in real-world scenarios.
	
	
\end{itemize}


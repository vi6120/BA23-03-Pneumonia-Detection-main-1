%%%%%%%%%%%%%%%%%%%%%%%%%%%
%
% $Autor: Sangram Patil$
% $Datum: 2023-03-17 11:15:45Z $
% $Pfad: GDV/Vortraege/latex - Ausarbeitung/Kapitel/MapleDateien.tex $
% $Version: 4732 $
%
%%%%%%%%%%%%%%%%%%%%%%%%%%%
%\VERSION{$ $Pfad: GDV/Vortraege/latex - Ausarbeitung/Kapitel/MapleDateien.tex $ $}{$ $Version: 4732 $ $}



\chapter{Deployment}
\section{User Interface}

In this Project for the deployment of our model, we are going to use WEB APPLICATION. the web application allows users to upload X-ray images and receive a classification result indicating whether the image shows signs of pneumonia. It allows patients to upload their X-ray images and tell them the results. 
For creating the framework for the web app, we will use Visual Studio where the user gives the input in the form of chest x-ray images and get the result as output.


The figure 8.1.: shows the road map for the web app. The user will upload an Image via a webpage and after submitting the image, the image will go to the Flask server where our trained model will be used to make predictions. After the predictions are made the response will be sent back to the webpage via the server.

\medskip

\begin{figure}
	\centering
	\begin{tikzpicture}
		\node at (0,0) {\includegraphics[width=\linewidth]{Images/USER MAP.png}};
	\end{tikzpicture}
	\caption{\textbf{Road Map For Web App}}
	\footnotesize \textbf{Reference:}\cite{Alhameed:2022}
	\label{fig:Road Map For Web App}
\end{figure}

\bigskip
\begin{figure}
	\centering
	\begin{tikzpicture}
		\node at (0,0) {\includegraphics[width=\linewidth]{Images/Interface}};
	\end{tikzpicture}
	\caption{\textbf{Web Page}}
	\footnotesize \textbf{Reference:}Author
	\label{fig:Web Page}
\end{figure}


\begin{itemize}
	\item	The figure 8.2.: provides an overview of how the website might appear. The user can upload the X-ray image here in the Choose file area as input.
	
\end{itemize}

\medskip




\begin{figure}
	\centering
	\begin{tikzpicture}
		\node at (0,0) {\includegraphics[width=\linewidth]{Images/Result}};
	\end{tikzpicture}
	\caption{\textbf{Result}}
	\footnotesize \textbf{Reference:} Author
	\label{fig:Result}
\end{figure}



\begin{itemize}
	\item	The data is sent to the server as it is provided to the web application. Any prediction made after that will be displayed as an output result. The figure 8.3.: displays a healthy outcome.
	
\end{itemize}


\bigskip

\section{Work flow}

The proposed workflow model of this network is demonstrated in figure 8.1.:, where firstly, the acquired labeled data are preprocessed, then split into 80:20, 70:30, and 60:40 ratios of test and train, applied with image augmentation properties. Furthermore, the images are subjected to various pre-trained models such as VGG16, VGG19, and Xception that involve transfer learning techniques. If the accuracy is not adequate in one ratio, the images are trained again with a different set of ratios. The models are chosen based on their accuracy and other evaluations.\cite{jain2022deep}

\begin{figure}
	\centering
	\begin{tikzpicture}
		\node at (0,0) {\includegraphics[width=\linewidth]{Images/Pneumonia detection Workflow.png}};
	\end{tikzpicture}
	\caption{\textbf{Deployment approach flow chart}}
	\footnotesize \textbf{Reference:}\cite{jain2022deep}
	\label{fig:Deployment Approach}
\end{figure}


\begin{table}[h]
	\centering
	\caption\textbf{Accuracy of proposed networks}
	\label{tab:Accuracy}
	\begin{tabular}{|c|c|c|c|}
		\hline
		Network & Training and Test Ratio & Training Acc. & Validation Acc. \\
		\hline
		CNN & 80-20 & 0.89 & 0.93 \\
		\cline{2-4}
		& 70-30 & 0.90 & 0.94 \\
		\cline{2-4}
		& 60-40 & 0.89 & 0.90 \\
		\hline
		Xception & 80-20 & 0.86 & 0.94 \\
		\cline{2-4}
		& 70-30 & 0.86 & 0.93 \\
		\cline{2-4}
		& 60-40 & 0.82 & 0.86 \\
		\hline
		VGG16 & 80-20 & 0.78 & 0.94 \\
		\cline{2-4}
		& 70-30 & 0.78 & 0.93 \\
		\cline{2-4}
		& 60-40 & 0.76 & 0.88 \\
		\hline
		VGG19 & 80-20 & 0.91 & 0.93 \\
		\cline{2-4}
		& 70-30 & 0.92 & 0.88 \\
		\cline{2-4}
		& 60-40 & 0.91 & 0.89 \\
		\hline
	\end{tabular}
	\footnotesize \textbf{Reference:}\cite{jain2022deep}
\end{table}

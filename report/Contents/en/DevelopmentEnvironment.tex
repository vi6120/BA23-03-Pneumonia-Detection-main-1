%%%
%
% $Autor: Vikas Ramaswamy$
% $Datum: 2023-06-17 $
% $Version:  1.0 $
%
% !TeX spellcheck = de_GB
%
%%%

\chapter{Development Environment: PyCharm}
\label{ch:pycharm}

We have chosen PyCharm as our development environment over other options like Visual Studio Code. PyCharm is a powerful integrated development environment (IDE) designed specifically for Python development. It provides a wide range of features to enhance productivity, code quality, and collaboration.

\section{Version}
We are using PyCharm Version 2023.1.3 (Build 231.9161.41) for our project.

\section{PyCharm Benefits}
Due to its many features and capabilities, PyCharm is a fantastic option for Python development, which is why we chose it. Throughout the development process, PyCharm provides a complete collection of tools and functions that increase productivity, code quality, and teamwork.

\section{Main Functions}
\begin{itemize}
	\item \textbf{Code Editing:} With syntax highlighting, code completion, formatting, and refactoring features, PyCharm offers a feature-rich code editor. It features sophisticated code analysis and supports different Python versions.
	\item \textbf{Debugging:} PyCharm's debugger allows developers to step through code, set breakpoints, inspect variables, and analyze program execution.
	\item \textbf{Code Navigation:} PyCharm enables easy navigation through code with features like Go to Definition, Find Usages, and Refactor.
	\item \textbf{Version Control Integration:} PyCharm seamlessly integrates with popular version control systems like Git, Mercurial, and SVN.
	\item \textbf{Testing and Profiling:} PyCharm has built-in support for running unit tests and offers profiling tools to analyze code performance.
	\item \textbf{Project Management:} PyCharm helps organize projects with templates, virtual environments, and easy package management.
\end{itemize}

\section{Subfunctions}

Each of the primary functions of PyCharm has a large number of subfunctions, including comprehensive code analysis options, a built-in terminal for running commands, interactive Python consoles, database connectivity, support for web development, and much more.

\section{Installation}
To install PyCharm, follow these steps:
\begin{enumerate}
	\item Visit the JetBrains website at \url{https://www.jetbrains.com/pycharm/}.
	\item Download the installer for your operating system.
	\item Run the installer and follow the on-screen instructions.
	\item Choose the edition (Professional or Community) and customize the installation settings if desired.
	\item Complete the installation process.
\end{enumerate}

\section{Configuration}
\subsection{General}
After installing PyCharm, you can configure general settings such as appearance, keymap, and editor behavior through the preferences or settings menu within the IDE.

\subsection{Special}
PyCharm also allows for specific configurations, including project interpreters, external tools, version control systems, code style, and plugin management.

\section{First Steps}
To start using PyCharm:
\begin{enumerate}
	\item Launch PyCharm after installation.
	\item Create a new project or import an existing one.
	\item Configure project settings like Python interpreter and virtual environments.
	\item Create a new Python file within the project.
	\item Write your Python code.
	\item Run the code using the Run button or associated keyboard shortcut.
	\item Observe the output in the console.
\end{enumerate}

\section{Program "Hello World"}
\subsection{Description}
A "Hello World" program is a simple introductory program that prints the phrase "Hello, World!" to the console. It is commonly used to verify the setup and functionality of a development environment.

\subsection{Manual}
To create a "Hello World" program in PyCharm:
\begin{enumerate}
	\item Open PyCharm and create a new project.
	\item Within the project, create a new Python file.
	\item In the Python file, write the following code: \texttt{print("Hello, World!")}
	\item Save the file.
	\item Run the program by clicking the Run button or using the associated keyboard shortcut.
	\item The output "Hello, World!" should be displayed in the console.
\end{enumerate}

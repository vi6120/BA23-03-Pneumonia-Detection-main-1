%%%%%
%%%%%%
%
% $Author: Sangram Patil $
% $Datum: 2023-06-30  $
% $Pfad: BA23-02-Sales-Predictor/manual/Chapter/Introduction.tex $
% $Version: 1.0 $
% $Reviewed by: Vikas And Adarsh $
% $Review Date: 2023-07-01 $
% $Description: Operation Of Pneumonia Detection
%%%%%%%%%%%%%%%%%%%%%%%%%%%%%%%%%%%%%%%%%%%%%%%%%%%%%%%%%%%%%%%%


\chapter{Help}

\section{Frequently Asked Questions}

\begin{enumerate}
	\item \textbf{How do I upload an image?}
	
To upload an image, click on the "Upload" button or drag and drop the image file into the designated area. Make sure the image file is in a compatible format, such as JPEG or PNG.
	
	\item \textbf{What should I do if my image is not uploading?}
	
Ensure that the image file size does not exceed the specified limit. If the problem persists, try using a different web browser or check your internet connection.
	
	\item \textbf{Can I upload multiple images at once?}
	
Currently, the web app supports the upload of one image at a time. However, you can repeat the process for multiple images individually.
	
	\item \textbf{How long does it take to get the prediction results?}
	
The prediction results are generated within a few seconds after the image is uploaded. The processing time may vary depending on the size of the image and the server's processing capacity.
	
	\item \textbf{How accurate are the prediction results?}
	
Our model has been trained on a large dataset and has shown high accuracy in pneumonia detection. However, it's important to note that no diagnostic tool is 100\% accurate, and consulting a medical professional is always recommended for accurate diagnosis.

	\item \textbf {How do I understand the outcome of the prediction?}
	
If the prediction result indicates "Pneumonia Detected," it means that the model has detected pneumonia in the uploaded image. If the result indicates "No Pneumonia Detected," it means that pneumonia is not detected. However, it's important to consult with a medical professional for an accurate diagnosis and further medical advice
	

\end{enumerate}

\section{Troubleshooting}

Troubleshooting is an essential aspect of maintaining an image processing web application for pneumonia detection. Here are some common issues that users may encounter and possible solutions to address them:


\begin{enumerate}
	\item \textbf{Image Upload Failure:}
	
     \begin{itemize}
     	\item Verify that the uploaded image file is in a supported format (e.g., JPEG, PNG) and meets the size restrictions set by the web app.
     	\item Check the internet connection to ensure a stable and reliable connection.
   		\item Clear the browser cache and try uploading the image again.
     	\item If the issue persists, provide an error message to the user with instructions on how to resolve the problem or contact support for further assistance.
     \end{itemize}
	
	\item \textbf{Slow Processing Time:}
	
	\begin{itemize}
		\item Check the server resources and ensure they are sufficient to handle the image processing workload.
	   \item Optimize the image processing pipeline by implementing efficient algorithms or techniques to speed up the analysis.
	  \item Inform the user about the estimated processing time to set expectations and reduce frustration.
		
	\end{itemize}
	
	\item \textbf{Inaccurate or Unreliable Results:}
	
\begin{itemize}
	\item 	Ensure that the deep learning model used for pneumonia detection is trained on a diverse, updated and high-quality dataset.
	\item 	Regularly update the model by incorporating new data and retraining it to improve accuracy.
	\item Implement cross-validation techniques to assess the model's performance and identify areas for improvement.
	\item 	Consider including a confidence score or probability value with the prediction results to indicate the level of certainty.
\end{itemize}
	
	\item \textbf{Compatibility Issues:}
	
\begin{itemize}
	\item 	Verify that the web application is compatible with different browsers and operating systems. Test the application on various platforms to identify and resolve any compatibility issues.
\item 	Check for any dependency conflicts and ensure that the required libraries and frameworks are up to date.
\item 	Provide clear instructions to users regarding the recommended browsers or system requirements for optimal performance.
\end{itemize}
	
	\item \textbf{Error Handling and Reporting:}
	
 \begin{itemize}
 	\item 	Implement comprehensive error handling mechanisms to catch and report any unexpected errors or exceptions.
 	\item 	Log detailed error messages to facilitate debugging and troubleshooting.
 	\item 	Display user-friendly error messages to guide users and provide instructions on how to resolve common issues.
 	\item 	Set up a system to collect user feedback and error reports, allowing you to identify recurring issues and address them promptly.
 \end{itemize}
	
	\item \textbf{User Support :}
	\begin{itemize}
		\item 	Provide a dedicated support channel, such as an email address or online chat, for users to reach out with their concerns or issues.
		\item 	Maintain up-to-date documentation and user guides that cover common troubleshooting scenarios and provide step-by-step instructions for resolving them.
		\item 	Communicate known issues or limitations clearly to users, along with any workarounds or alternative solutions.
		
	\end{itemize}
\end{enumerate}












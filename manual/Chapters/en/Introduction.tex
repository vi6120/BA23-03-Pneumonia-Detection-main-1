%%%%%%
%
% $Author: Sangram Patil $
% $Datum: 2023-06-30  $
% $Pfad: BA23-02-Sales-Predictor/manual/Chapter/Introduction.tex $
% $Version: 1.0 $
% $Reviewed by: Vikas And Adarsh $
% $Review Date: 2023-07-01 $
% $Description: Introduction about what is Pneumonia Detection
%
\chapter{Getting to know about Pneumonia Detection }

\textbf{What is Pneumonia Detection?}


Pneumonia is a respiratory infection caused by bacteria or viruses; it affects many individuals, especially in developing and underdeveloped nations, where high levels of pollution, unhygienic living conditions, and overcrowding are relatively common, together with inadequate medical infrastructure. Pneumonia causes pleural effusion, a condition in which fluids fill the lung, causing respiratory difficulty. Early diagnosis of pneumonia is crucial to ensure curative treatment and increase survival rates. Chest X-ray imaging is the most frequently used method for diagnosing pneumonia. However, the examination of chest X-rays is a challenging task and is prone to subjective variability. In this study, we developed a computer-aided diagnosis system for automatic pneumonia detection using chest X-ray images.


Pneumonia detection involves identifying the presence of pneumonia, which is an infection or inflammation of the lungs. It typically includes a combination of clinical assessment, medical imaging (such as chest X-rays or CT scans), laboratory tests (such as blood tests and sputum cultures), and symptom evaluation. The goal is to identify characteristic symptoms, abnormal lung sounds, and imaging findings that indicate the presence of pneumonia. Additionally, the analysis of patient data, including demographics, medical history, and vital signs, helps healthcare professionals in making an accurate diagnosis and determining appropriate treatment options.


Pneumonia detection involves analyzing medical images, such as chest X-rays or CT scans, to identify signs of pneumonia. Here are some key features commonly used in Pneumonia Detection:
\begin{itemize}
	\item \textbf{Image Preprocessing:} Prior to analysis, the medical images undergo preprocessing steps to enhance the relevant features and remove noise or artifacts. This may involve resizing, cropping, normalization, or applying filters to improve image quality.
	
	\item \textbf{Lung Segmentation:} In order to focus the analysis on the lung area, lung segmentation is performed to separate the lung region from the rest of the image. This step helps isolate potential pneumonia regions for further examination.
	
	\item \textbf{Chest X-ray:} Imaging plays a crucial role in pneumonia detection. Chest X-rays can help visualize the lungs and identify areas of infection or inflammation. Typical findings include infiltrates (consolidation or patchy opacities) in the affected lung regions.
	
	\item \textbf{Artificial Intelligence (AI) Algorithms:} The app can employ AI algorithms, such as machine learning or deep learning models, to aid in pneumonia detection. These algorithms can be trained on large datasets of pneumonia images to enhance accuracy and speed in identifying potential cases.
	
	\item \textbf{Accessibility and Security:} Accessible to ensure it can be used on various devices and browsers. Additionally, robust security measures should be implemented to protect patient data and comply with privacy regulations.
	

\end{itemize}


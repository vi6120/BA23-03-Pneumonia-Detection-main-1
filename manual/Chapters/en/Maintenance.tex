%%%%%%
%%%%%%
%
% $Author: Sangram Patil $
% $Datum: 2023-06-30  $
% $Pfad: BA23-02-Sales-Predictor/manual/Chapter/Introduction.tex $
% $Version: 1.0 $
% $Reviewed by: Vikas And Adarsh $
% $Review Date: 2023-07-01 $
% $Description: Maintenance
%%%%%%%%%%%%%%%%%%%%%%%%%%%%%%%%%%%%%%%%%%%%%%%%%%

\chapter{Maintenance for Pneumonia Detection Web Application}

As the online realm is ever-evolving, the Pneumonia Detection Project demands periodic maintenance and frequent upgrades to function optimally. An accurate and efficient operation of a web app for pneumonia Detection depends on routine upkeep and upgrades. Here are some important factors to be consider for maintaining a  Pneumonia Detection web app.

\begin{enumerate}
	\item \textbf{	Model Updates: }
	
	 Maintain a close eye on developments in algorithms and research for the Pneumonia Detection in order to implement improvements and raise prediction accuracy. To keep the model performing well, periodically retrain it using new data.
	
	\item \textbf{	Dataset Updates: }
	
	Ensure the dataset used for training the model is regularly updated. Incorporate new data that represents a diverse range of pneumonia cases to improve the model's ability to detect different variations of pneumonia accurately.
	
	\item \textbf{Bug Fixes and Performance Optimization:} 
	
	 Continuously monitor and address any bugs or issues reported by users. Conduct regular testing and performance optimization to enhance the web app's speed, efficiency, and user experience.
	
	\item \textbf{Security Updates: } 
	
Stay up to date with the latest security protocols and best practices to safeguard user data and protect against potential security vulnerabilities. Maintain the web app's safety measure by implementing the essential security protections and upgrades.
	
	\item \textbf{Compatibility and Dependency Updates:} 
	
 Keep track of any changes or updates to the dependencies, libraries, or frameworks used in the web app. Regularly update these components to ensure compatibility with the latest versions and maintain a stable environment.
	
	\item \textbf{Backup and Recovery:}
	
Implement a reliable backup plan to protect user data and ensure its recoverability in case of any unforeseen events or data loss. Regularly back up the relevant data and establish recovery procedures.
	
	\item \textbf{	User Feedback and Support:} 
	
 Maintain a support system to address user queries and provide assistance as needed related to Pneumonia Detection. Incorporate user feedback to enhance the web app's usability and features.
 
 	\item \textbf{Documentation and User Guides: }
 	
 	Keep the documentation and user guides up to date to reflect any changes or updates made to Pneumonia Detection Web App . Ensure that users have access to accurate and relevant information on how to use the app effectively.
 
 
\end{enumerate}



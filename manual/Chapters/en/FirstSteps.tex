%%%%%%
%%%%%%
%
% $Author: Sangram Patil $
% $Datum: 2023-06-30  $
% $Pfad: BA23-02-Sales-Predictor/manual/Chapter/Introduction.tex $
% $Version: 1.0 $
% $Reviewed by: Vikas And Adarsh $
% $Review Date: 2023-07-01 $
% $Description: Operation Of Pneumonia Detection
%%%%%%%%%%%%%%%%%%%%%%%%%%%%%%%%%%%%%%%%%%%%%%%%%%%%%%%%%%%%%%%%

\chapter{First Step in Pneumonia Detection}

\section{Interface for Pneumonia Detection:}

\begin{figure}[h!]
	\centering
	\includegraphics[width=\textwidth]{Images/Interface}
	\caption{Interface Of Pneumonia Detection}
\end{figure}

\section{Steps for Pneumonia Detection}

\begin{enumerate} [label=Step \arabic*:]
	\item \textbf {Access the Web Application:}
	
	Open a web browser and enter the URL or click on the provided link to access the pneumonia detection web application.
		\begin{figure}[h!]
			\centering
			\includegraphics[width=\textwidth]{Images/User Interface}
			\caption{User Interface}
		\end{figure}
	
	\item \textbf {Upload Chest X-ray Image:}
	
	Click on the “Choose File” button to open a file selection dialog box. 
		\begin{figure}[h!]
			\centering
			\includegraphics[width=\textwidth]{Images/Upload}
			\caption{Choose File Option}
		\end{figure}
	
	\item \textbf {Select the Chest X-ray Image:} 
	
	In the file selection dialog box, navigate to the location where the chest X-ray image is stored on your device. Select the desired image file and click on the "Open".
		\begin{figure}[h!]
			\centering
			\includegraphics[width=\textwidth]{Images/File}
			\caption{Selection Of File}
		\end{figure}
	
	\item \textbf {Image Processing and Analysis:}
	
	After selecting the image click on the “Submit”. After the image is uploaded, the web application will start processing. This process may take a few seconds to complete.
		\begin{figure}[h!]
			\centering
			\includegraphics[width=\textwidth]{Images/Submit}
			\caption{Submit Option}
		\end{figure}
	
	\item \textbf {View Pneumonia Prediction:} 
	
	Once the analysis is finished, the web application will present the pneumonia prediction result to you as shown in Figure.
		\begin{figure}[h!]
			\centering
			\includegraphics[width=0.5\textwidth]{Images/Result}
			\caption{Result Display}
		\end{figure}	
	
	\item \textbf {Interpret the Result:}
	
	There are two possibility that is “Person in Healthy” or “Person has Pneumonia”
	\begin{itemize}
		
	
	\item \textbf {Positive Result:}
	
	 For this, massage is displayed as “Person has Pneumonia”.As shown in fig 3.7
		\begin{figure}[h!]
			\centering
			\includegraphics[width=0.5\textwidth]{Images/Positive}
			\caption{Positive Result}
		\end{figure}
	
	
	
	\item \textbf {Negative Result:}
	
	 For this the, massage is displayed as “Person in Healthy”. As shown in fig. 3.8
		\begin{figure}[h!]
			\centering
			\includegraphics[width=0.5\textwidth]{Images/Negative}
			\caption{Negative Result}
		\end{figure}
	
\end{itemize}


		
\end{enumerate}



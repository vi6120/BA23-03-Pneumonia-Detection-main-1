%%%%%%
%%%%%%
%
% $Author: Sangram Patil $
% $Datum: 2023-06-30  $
% $Pfad: BA23-02-Sales-Predictor/manual/Chapter/Introduction.tex $
% $Version: 1.0 $
% $Reviewed by: Vikas And Adarsh $
% $Review Date: 2023-07-01 $
% $Description: Operation Of Pneumonia Detection
%%%%%%%%%%%%%%%%%%%%%%%%%%%%%%%%%%%%%%%%%%%%%%%%%%%%%%%%%%%%%%%%

\chapter{Operation Of Pneumonia Detection}

\section{Software Setup}
To set up the software for pneumonia detection, you can follow these steps:
\begin{enumerate}
	\item \textbf {Download and install Python:}
	
	Ensure that Python is installed on your system. You can download the latest version of Python from the official Python website and follow the installation instructions specific to your operating system. Visit the official Python website (\texttt{https://www.python.org}) and download the latest version of Python for your operating system.
	
	\item \textbf {Install PyCharm:}
	
	 Visit the JetBrains website (\texttt{https://www.jetbrains.com/pycharm}) and download PyCharm Community Edition, which is the free version. Install PyCharm by following the installation instructions specific to your operating system.
	
	\item \textbf {Install Required Libraries:}
	 
	Use pip, the Python package manager, to install the necessary libraries and frameworks for pneumonia detection. Some essential libraries for image processing and machine learning in this context may include OpenCV, Pillow, TensorFlow, Keras.
	
	\item \textbf {Set Up Flask Web Framework:}
	
	 Install the Flask web framework, which will be used to develop the web application. Use pip to install Flask
	 
	 \item \textbf{Train Pneumonia Detection Model:}
	 
	 Train your model using labeled data. If you train your model, ensure that the training data is properly prepared and augmented, and follow standard machine learning practices for training, validation, and evaluation.
	 
	  \item \textbf {Build the Web Application:} 
	  
	  Develop the web application using Flask and the necessary HTML, CSS, and JavaScript files. Define routes, handle user requests, and integrate the pneumonia detection model into the application's backend.
	 
	\item \textbf { Test the Application:}
	
	Test the functionality of the web application to ensure that it correctly processes uploaded images, performs pneumonia detection, and provides accurate results. Use sample images or a test dataset to verify the performance of the application.
	
	\item \textbf {Deployment:}
	
	Choose a suitable platform or hosting service to deploy the web application. Popular options include cloud platforms like Google Cloud Platform (GCP)
	
	
\end{enumerate}

\subsection{System Requirements }

\begin{itemize}
	
	\item \textbf {Software requirements:}
	
	\begin{table}[h]
		\centering
		\caption{Software requirements}
		\label{tab:Software requirements}
		\begin{tabular}{|c|c|c|}
			\hline
			\textbf{Software/Package} & \textbf{Open Source/License} & \textbf{Version}\\
			\hline
			VS Code & Open Source & 1.64.2\\
			\hline
			Anaconda-Navigator & Open Source & 2.1.4\\
			\hline
			Conda & Open Source & 4.11.0\\
			\hline
			Keras & Open Source & 2.8.0\\
			\hline
			TensorFlow & Open Source & 2.8.0\\
			\hline
		\end{tabular}
	%	\footnotesize \textbf{Reference:}Author
	\end{table}


\item \textbf {Hardware requirements:}


\begin{table}[h]
	\centering
	\caption{Hardware requirements}
	\label{tab:Hardware requirementss}
	\begin{tabular}{|c|c|c|}
		\hline
		\textbf{Hardware Name} & \textbf{Description} & \textbf{Quantity}\\
		\hline
		Computer Display & 1080p (1920x1080) resolution or higher & 1\\
		\hline
		Keyboard & Standard American/German language & 1\\
		\hline
		Mouse & Standard 3 button wired or wireless & 1\\
		\hline
		Laptop & Capable of running deep learning algorithms & 1\\
		\hline
	\end{tabular}
%	\footnotesize \textbf{Reference:}Author
\end{table}

\end{itemize}

%%%%%%
%%%%%%
%
% $Author: Sangram Patil $
% $Datum: 2023-06-30  $
% $Pfad: BA23-02-Sales-Predictor/manual/Chapter/Introduction.tex $
% $Version: 1.0 $
% $Reviewed by: Vikas And Adarsh $
% $Review Date: 2023-07-01 $
% $Description: Functions Of Pneumonia Detection
%%%%%%%%%%%%%%%%%%%%%%%%%%%%%%%%%%%%%%%%%%%%%%%%%%

\chapter{Functions of Pneumonia Detection Interface}

\textbf{Functions of Pneumonia Detection Interface}

\begin{enumerate}
	\item \textbf{Choose File:} 
	
	To start the analysis for the detection of pneumonia, click "Choose File" to upload your chest X-ray image. It also displays information about the file's selection state as shown in the figure.
	
	
	\begin{figure}[h!]
		\centering
		\includegraphics[width=\textwidth]{Images/CHOOSEFILE }
		\caption{Coose File Function }
	\end{figure}

	
	\item \textbf{	Submit:}
	
	 The processing and analysis of the submitted image for the identification of pneumonia are started using the "Submit" feature. The user can view the results of the prediction by clicking the "Submit" button, which instructs the web app to process the image according to the trained model.
	 
	 
	 \begin{figure}[h!]
	 	\centering
	 	\includegraphics[width=\textwidth]{Images/Submit}
	 	\caption{ Submit Function }
	 \end{figure}
	
	\item \textbf{Go Back Functions}: 
	
	By clicking the "Go back" button , the user can navigate back to a previous screen. Where user can upload the other photo for analysis. 
	
	 \begin{figure}[h!]
		\centering
		\includegraphics[width=\textwidth]{Images/Goback}
		\caption{ Go back Function }
	\end{figure}
	
	
	\item \textbf{Result Visualization:} 
	
	The result visualization function is responsible for displaying the prediction results to the user in a clear and informative manner. After the image processing and analysis are completed, the result visualization function presents the prediction outcome, indicating whether pneumonia is detected or not.
	
	\begin{itemize}
		\item Image : In this the uploaded image is showed
		\item	Massage: Massage indicated the prediction result like “Person is Healthy” or “Person has pneumonia”	
	\end{itemize}
	
	 \begin{figure}[h!]
		\centering
		\includegraphics[width=0.5\textwidth]{Images/Result}
		\caption{ Result Visualization }
	\end{figure}
	
	
	
	
\end{enumerate}

By providing accurate and relevant information for each input parameter, you enable the model to generate more precise sales predictions tailored to your specific requirements and context.
